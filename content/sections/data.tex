 % -*- root: ../../fss.tex -*-
\section{Data Collection}

\begin{frame}
    \frametitle{How to gather Sentiment Variables?}
    \scalebox{0.78}{
         % -*- root: FSS3.tex -*-
 \label{flowchart}
\begin{tikzpicture}[node distance=2cm]
\node (articles) [doc] {\hyperlink{lab:articles}{Articles}};
\node (scraping) [standard, right of=articles, xshift=0.7cm] {\hyperlink{lab:scraping}{Scraping}};
\node (nlp) [standard, right of=scraping, xshift=0.5cm] {\hyperlink{lab:nlp}{NLP}};
\node (projection) [standard, right of=nlp, xshift=0.7cm] {\hyperlink{lab:projection}{Projection}};

\node (sentiment) [doc, right of=projection, xshift=1.2cm] {Sentiment};

\node[bigbox, below of=scraping, yshift=-0.7cm, xshift=-1.3cm](outbox)
{
%    \begin{minipage}{1.5cm}
%    \center
%        \tikz{\node[box]{\scriptsize URL};}
%        \tikz{\node[box, yshift=1cm]{\scriptsize Author};}
%        \tikz{\node[box]{\scriptsize Symbol};}
%        \tikz{\node[box]{\scriptsize Date};}
%        \tikz{\node[box]{\scriptsize Text};}
%    \end{minipage}

    \begin{minipage}{1.5cm}
    \center
    \begin{tikzpicture}[node distance=0.55cm]
    \node (url) [box] {\scriptsize URL};
    \node (author) [box, below of=url] {\scriptsize Author};
    \node (symbol) [box, below of=author] {\scriptsize Symbol};
    \node (date) [box, below of=symbol] {\scriptsize Date};
    \node (text) [box, below of=date] {\scriptsize Text};
    \end{tikzpicture}
    \end{minipage}
};

\node (nasdaq) [doc, below of=scraping, xshift=-1.3cm, yshift=-3cm] {\hyperlink{lab:articles}{Nasdaq Articles}};
\node (rdc) [below of=scraping, xshift=-1.3cm, yshift=-4cm] {
\href{http://sfb649.wiwi.hu-berlin.de/fedc/index.php}{\includegraphics[scale=0.18]{img/logos/rdc}}

};

\node (rdc2) [below of=scraping, xshift=-1.3cm, yshift=-4cm] {};

\node[bigbox, below of=nlp, yshift=-0.25cm](outbox2)
{
%    \begin{minipage}{1.5cm}
%    \center
%        \tikz{\node[box]{\scriptsize URL};}
%        \tikz{\node[box, yshift=1cm]{\scriptsize Author};}
%        \tikz{\node[box]{\scriptsize Symbol};}
%        \tikz{\node[box]{\scriptsize Date};}
%        \tikz{\node[box]{\scriptsize Text};}
%    \end{minipage}

    \begin{minipage}{1.6cm}
    \center
    \begin{tikzpicture}[node distance=0.55cm]
    \node (token) [box2] {\scriptsize Token};
    \node (negation) [box2, below of=token] {\scriptsize Negation};
    \node (pos) [box2, below of=negation] {\scriptsize POS};
    \node (lemmatization) [box2, below of=pos] {\scriptsize Lemmata};
    \end{tikzpicture}
    \end{minipage}
};
%\node (nasdaq) [standard, below of=articles, yshift=5cm] {Scraping};

\node[bigbox, below of=projection, yshift=-0.30cm, xshift=1.3cm](outbox3)
{
%    \begin{minipage}{1.5cm}
%    \center
%        \tikz{\node[box]{\scriptsize URL};}
%        \tikz{\node[box, yshift=1cm]{\scriptsize Author};}
%        \tikz{\node[box]{\scriptsize Symbol};}
%        \tikz{\node[box]{\scriptsize Date};}
%        \tikz{\node[box]{\scriptsize Text};}
%    \end{minipage}

    \begin{minipage}{2cm}
    \center
    \begin{tikzpicture}[node distance=0.55cm]
    \node [box3, yshift=0.80cm]{\scriptsize Unsupervised};
    \node (bl) [box2] {\scriptsize BL};
    \node (lm) [box2, below of=token] {\scriptsize LM};
    \end{tikzpicture}
    \end{minipage}
};

\node[bigbox2, below of=projection, yshift=-2.5cm, xshift=1.3cm](outbox4)
{
%    \begin{minipage}{1.5cm}
%    \center
%        \tikz{\node[box]{\scriptsize URL};}
%        \tikz{\node[box, yshift=1cm]{\scriptsize Author};}
%        \tikz{\node[box]{\scriptsize Symbol};}
%        \tikz{\node[box]{\scriptsize Date};}
%        \tikz{\node[box]{\scriptsize Text};}
%    \end{minipage}

    \begin{minipage}{2cm}
    \center
    \begin{tikzpicture}[node distance=0.55cm]
    \node [box3, yshift=0.80cm]{\scriptsize Supervised};
    \node (sm) [box2] {\scriptsize SM};
    \end{tikzpicture}
    \end{minipage}
};

\node[box3, below of=sentiment, yshift=0.5cm, xshift=0.4cm](happy2){
};
\node[box3, below of=sentiment, yshift=0.5cm, xshift=0.6cm](happy){
    \includegraphics[width=15pt]{img/icons/happy}
};

\node[box3, below of=sentiment, yshift=-0.2cm, xshift=0.4cm](neutral2){
};
\node[box3, below of=sentiment, yshift=-0.2cm, xshift=0.6cm](neutral){
    \includegraphics[width=15pt]{img/icons/neutral}
};

\node[box3, below of=sentiment, yshift=-0.9cm, xshift=0.4cm](sad2){
};
\node[box3, below of=sentiment, yshift=-0.9cm, xshift=0.6cm](sad){
    \includegraphics[width=15pt]{img/icons/sad}
};


\draw [arrow] (articles) -- (scraping);
\draw [arrow] (scraping) -- (nlp);
\draw [arrow] (nlp) -- (projection);
\draw [arrow] (projection) -- (sentiment);
\draw [line width=0.25mm] (scraping) |- (outbox);
\draw [line width=0.25mm] (nlp) -- (outbox2);
\draw [arrow] (outbox) -- (nasdaq);
\draw [arrow] (nasdaq) -- (rdc2);
\draw [line width=0.25mm] (projection) |- (outbox3);
\draw [line width=0.25mm] (projection) |- (outbox4);
\draw [line width=0.25mm] (sentiment) |- (happy2);
\draw [line width=0.25mm] (sentiment) |- (neutral2);
\draw [line width=0.25mm] (sentiment) |- (sad2);

%\node (in1) [doc, below of=start] {Input};
%\node (pro1) [process, below of=in1] {Process 1};
%\node (dec1) [decision, below of=pro1] {Decision 1};
\end{tikzpicture}
    }
\end{frame}


\begin{frame}
    \frametitle{Nasdaq Articles}
    \label{lab:articles}

    \begin{itemize}
        \item Terms of Service permit web scraping
        \item Currently $>580$k articles between October 2009 and December 2017
        \item Data available at \hspace{-0.1pt}
        \textcolor{iseblue}{\href{http://sfb649.wiwi.hu-berlin.de/fedc/index.php}{\includegraphics[scale=0.18]{img/logos/rdc}}}
    \end{itemize}
\end{frame}


% updated 20170605
\begin{frame}
    \frametitle{Article Timeline}
    \vspace{5pt}
    \begin{figure}[htb]
        \begin{center}
            \includegraphics[scale=0.66]{img/figures/n_articles}
        \end{center}
        \vspace{-10pt}
        \caption{Number of Sector-specific Articles per Day}
    \end{figure}
\end{frame}



\begin{frame}
\frametitle{Attention Ratio}

Attention ratio by \textcolor{iseblue}{Zhang et al (2016)} \\
\begin{equation}
\textcolor{iseblue}{
AR_{i} = \textstyle T^{-1} \sum_{t=1}^T{\operatorname{\mathbf{I}}\big( n_{i, t} > 0 \big) }
}
\end{equation}
with $n$ as the number of published articles for company $i$ on day $t$.
\vspace{10pt}
% table created with chapter_data/table_100_companies.R
\begin{table}[ht]
\centering
\begin{tabular}{r|rrrrrr}
  \hline
  \hline
Quantile & 0\% & 20\% & 40\% & 60\% & 80\% & 100\% \\ 
Attention Ratio & 0.01 & 0.18 & 0.22 & 0.30 & 0.44 & 0.99 \\ 
   \hline
  \hline
\end{tabular}
    \caption{Quantiles of Attention Ratio for all Nasdaq Companies}
    \label{tab:ar_total}
\end{table}
\vspace{-25pt}
\begin{itemize}
    \item Media coverage differs between companies
    \item Improve signal to noise ratio by pre-selection of 100 companies

\end{itemize}
\end{frame}

\begin{frame}
\frametitle{Attention Ratio II}
\begin{table}[ht]
\scalebox{0.9}{
\centering
\begin{tabular}{l|rrrrr}
  \hline
  \hline
 &  \multicolumn{5}{c}{Attention Ratio}  \\ 
  Sector & Min & Q1 & Q2 & Q3 & Max  \\ 
  \hline
  Consumer Discretionary &  0.448 &  0.523 &  0.630 &  0.737 &  0.929  \\ 
  Consumer Staples &  0.443 &  0.500 &  0.521 &  0.622 &  0.871 \\ 
  Energy &  0.448 &  0.512 &  0.534 &  0.697 &  0.854  \\ 
  Financials &  0.464 &  0.616 &  0.686 &  0.891 &  0.979  \\ 
  Health Care &  0.443 &  0.512 &  0.583 &  0.636 &  0.841  \\ 
  Industrials &  0.458 &  0.522 &  0.577 &  0.661 &  0.857  \\ 
  Information Technology &  0.444 &  0.528 &  0.655 &  0.848 &  0.991  \\ 
  Materials &  0.533 &  0.585 &  0.637 &  0.640 &  0.643  \\ 
  Telecommunication Services &  0.871 &  0.885 &  0.899 &  0.913 &  0.927  \\ 
  Utilities &  0.463 &  0.463 &  0.463 &  0.463 &  0.463  \\ 
  \hline
  \hline
\end{tabular}
}
\caption{Attention Ratio of 100 Companies by Sector. Q1, Q2 and Q3 represent 25\%, 50\% and 75\% quantile, respectively.}
\end{table}
\end{frame}


\begin{frame}
\frametitle{Sector-specific articles}

    \begin{table}[ht]
        \centering
        \scalebox{0.9}{
        \begin{tabular}{l|crr}
            \hline \hline
            Sector & Abbr.&\# Articles & \# Comp. \\ 
            \hline
            Consumer Discretionary     & CD & 30,360 &  19 \\ 
            Consumer Staples           & CS & 12,210 &  10 \\ 
            Energy                     & EN & 10,410 &   8 \\ 
            Financials                 & FI & 34,570 &  13 \\ 
            Health Care                & HC & 16,950 &  13 \\ 
            Industrials                & IN & 16,440 &  13 \\ 
            Information Technology     & IT & 44,120 &  18 \\ 
            Materials                  & MA & 3,820 &   3 \\ 
            Telecommunication Services & TE & 5,880  &   2 \\ 
            Utilities                  & UT & 780  &   1 \\ 
            \hline \hline
        \end{tabular}
        }
        \caption{Number of Articles per Sector between 10/2009 and 12/2017}
    \end{table}
\end{frame}
